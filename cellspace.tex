% (C) 2006-2019 Josselin Noirel
% 
% Package cellspace: Ensuring a minimal spacing of table cells.
%
% This material is subject to the LaTeX Project Public License. See
% http://www.ctan.org/tex-archive/help/Catalogue/licenses.lppl.html .
%
\def \filedate    {2019/03/11}%
\def \fileversion {v1.8.1}%

\ProvidesFile{cellspace}%
  [\filedate \space \fileversion \space cellspace documentation (JN)]
\pdfoutput = 1
\documentclass[12pt]{ltxdoc}

\usepackage{ifpdf}
\ifpdf
  \PassOptionsToPackage{pdftex}{geometry}
\else
  \PassOptionsToPackage{dvips}{geometry}
\fi
\usepackage[ascii]{inputenc}
\usepackage[T1]{fontenc}
\usepackage{amsmath}
\usepackage{textcomp}
\usepackage{lmodern}
\usepackage{microtype}
\usepackage[a4paper, margin = 1.3in]{geometry}
\usepackage[math]{cellspace}
\usepackage{booktabs}
\usepackage{tabularx}
\usepackage{moreverb}
\usepackage{floatrow}
  \floatsetup{margins = centering}
\usepackage{algorithmic}
\usepackage[british]{babel}

% -----------------------------------------------------------------

\newcommand*{\compare}[2]{%
  \par \begingroup \trivlist \item \centering
%%   \parbox[t]{0.5\linewidth}{#1}%
%%   \parbox[t]{0.5\linewidth}{#2}%
    \vskip-\baselineskip
  \makebox[0pt]{%
    \mbox{#1}%
    \quad
    \mbox{#2}%
  }
  \endtrivlist \endgroup \par
}
\renewcommand*{\meta}[1]{\textnormal{\textlangle \emph{#1}\textrangle}}
\newcommand*{\pack}{\textsf}
\newcommand*{\tex}[1]{\texttt{#1}}
\renewcommand*{\cmd}[1]{\tex{\textbackslash #1}}
\makeatletter
\newenvironment*{showexample}{%
  \par \vskip-\baselineskip
  \bgroup \obeylines
  \@showexample
}{%
    \endverbatimwrite
  \endgroup
  \begingroup
    \footnotesize
    \samepage
    \verbatimtabinput{cellspace.tmp}%
  \endgroup
  \nobreak \@nobreaktrue \vskip-\baselineskip \vskip\z@skip
  \begingroup \@nameuse{\@showexampleenv}%
    \small
    % (C) 2006-2019 Josselin Noirel
% 
% Package cellspace: Ensuring a minimal spacing of table cells.
%
% This material is subject to the LaTeX Project Public License. See
% http://www.ctan.org/tex-archive/help/Catalogue/licenses.lppl.html .
%
\def \filedate    {2019/03/11}%
\def \fileversion {v1.8.1}%

\ProvidesFile{cellspace}%
  [\filedate \space \fileversion \space cellspace documentation (JN)]
\pdfoutput = 1
\documentclass[12pt]{ltxdoc}

\usepackage{ifpdf}
\ifpdf
  \PassOptionsToPackage{pdftex}{geometry}
\else
  \PassOptionsToPackage{dvips}{geometry}
\fi
\usepackage[ascii]{inputenc}
\usepackage[T1]{fontenc}
\usepackage{amsmath}
\usepackage{textcomp}
\usepackage{lmodern}
\usepackage{microtype}
\usepackage[a4paper, margin = 1.3in]{geometry}
\usepackage[math]{cellspace}
\usepackage{booktabs}
\usepackage{tabularx}
\usepackage{moreverb}
\usepackage{floatrow}
  \floatsetup{margins = centering}
\usepackage{algorithmic}
\usepackage[british]{babel}

% -----------------------------------------------------------------

\newcommand*{\compare}[2]{%
  \par \begingroup \trivlist \item \centering
%%   \parbox[t]{0.5\linewidth}{#1}%
%%   \parbox[t]{0.5\linewidth}{#2}%
    \vskip-\baselineskip
  \makebox[0pt]{%
    \mbox{#1}%
    \quad
    \mbox{#2}%
  }
  \endtrivlist \endgroup \par
}
\renewcommand*{\meta}[1]{\textnormal{\textlangle \emph{#1}\textrangle}}
\newcommand*{\pack}{\textsf}
\newcommand*{\tex}[1]{\texttt{#1}}
\renewcommand*{\cmd}[1]{\tex{\textbackslash #1}}
\makeatletter
\newenvironment*{showexample}{%
  \par \vskip-\baselineskip
  \bgroup \obeylines
  \@showexample
}{%
    \endverbatimwrite
  \endgroup
  \begingroup
    \footnotesize
    \samepage
    \verbatimtabinput{cellspace.tmp}%
  \endgroup
  \nobreak \@nobreaktrue \vskip-\baselineskip \vskip\z@skip
  \begingroup \@nameuse{\@showexampleenv}%
    \small
    % (C) 2006-2019 Josselin Noirel
% 
% Package cellspace: Ensuring a minimal spacing of table cells.
%
% This material is subject to the LaTeX Project Public License. See
% http://www.ctan.org/tex-archive/help/Catalogue/licenses.lppl.html .
%
\def \filedate    {2019/03/11}%
\def \fileversion {v1.8.1}%

\ProvidesFile{cellspace}%
  [\filedate \space \fileversion \space cellspace documentation (JN)]
\pdfoutput = 1
\documentclass[12pt]{ltxdoc}

\usepackage{ifpdf}
\ifpdf
  \PassOptionsToPackage{pdftex}{geometry}
\else
  \PassOptionsToPackage{dvips}{geometry}
\fi
\usepackage[ascii]{inputenc}
\usepackage[T1]{fontenc}
\usepackage{amsmath}
\usepackage{textcomp}
\usepackage{lmodern}
\usepackage{microtype}
\usepackage[a4paper, margin = 1.3in]{geometry}
\usepackage[math]{cellspace}
\usepackage{booktabs}
\usepackage{tabularx}
\usepackage{moreverb}
\usepackage{floatrow}
  \floatsetup{margins = centering}
\usepackage{algorithmic}
\usepackage[british]{babel}

% -----------------------------------------------------------------

\newcommand*{\compare}[2]{%
  \par \begingroup \trivlist \item \centering
%%   \parbox[t]{0.5\linewidth}{#1}%
%%   \parbox[t]{0.5\linewidth}{#2}%
    \vskip-\baselineskip
  \makebox[0pt]{%
    \mbox{#1}%
    \quad
    \mbox{#2}%
  }
  \endtrivlist \endgroup \par
}
\renewcommand*{\meta}[1]{\textnormal{\textlangle \emph{#1}\textrangle}}
\newcommand*{\pack}{\textsf}
\newcommand*{\tex}[1]{\texttt{#1}}
\renewcommand*{\cmd}[1]{\tex{\textbackslash #1}}
\makeatletter
\newenvironment*{showexample}{%
  \par \vskip-\baselineskip
  \bgroup \obeylines
  \@showexample
}{%
    \endverbatimwrite
  \endgroup
  \begingroup
    \footnotesize
    \samepage
    \verbatimtabinput{cellspace.tmp}%
  \endgroup
  \nobreak \@nobreaktrue \vskip-\baselineskip \vskip\z@skip
  \begingroup \@nameuse{\@showexampleenv}%
    \small
    % (C) 2006-2019 Josselin Noirel
% 
% Package cellspace: Ensuring a minimal spacing of table cells.
%
% This material is subject to the LaTeX Project Public License. See
% http://www.ctan.org/tex-archive/help/Catalogue/licenses.lppl.html .
%
\def \filedate    {2019/03/11}%
\def \fileversion {v1.8.1}%

\ProvidesFile{cellspace}%
  [\filedate \space \fileversion \space cellspace documentation (JN)]
\pdfoutput = 1
\documentclass[12pt]{ltxdoc}

\usepackage{ifpdf}
\ifpdf
  \PassOptionsToPackage{pdftex}{geometry}
\else
  \PassOptionsToPackage{dvips}{geometry}
\fi
\usepackage[ascii]{inputenc}
\usepackage[T1]{fontenc}
\usepackage{amsmath}
\usepackage{textcomp}
\usepackage{lmodern}
\usepackage{microtype}
\usepackage[a4paper, margin = 1.3in]{geometry}
\usepackage[math]{cellspace}
\usepackage{booktabs}
\usepackage{tabularx}
\usepackage{moreverb}
\usepackage{floatrow}
  \floatsetup{margins = centering}
\usepackage{algorithmic}
\usepackage[british]{babel}

% -----------------------------------------------------------------

\newcommand*{\compare}[2]{%
  \par \begingroup \trivlist \item \centering
%%   \parbox[t]{0.5\linewidth}{#1}%
%%   \parbox[t]{0.5\linewidth}{#2}%
    \vskip-\baselineskip
  \makebox[0pt]{%
    \mbox{#1}%
    \quad
    \mbox{#2}%
  }
  \endtrivlist \endgroup \par
}
\renewcommand*{\meta}[1]{\textnormal{\textlangle \emph{#1}\textrangle}}
\newcommand*{\pack}{\textsf}
\newcommand*{\tex}[1]{\texttt{#1}}
\renewcommand*{\cmd}[1]{\tex{\textbackslash #1}}
\makeatletter
\newenvironment*{showexample}{%
  \par \vskip-\baselineskip
  \bgroup \obeylines
  \@showexample
}{%
    \endverbatimwrite
  \endgroup
  \begingroup
    \footnotesize
    \samepage
    \verbatimtabinput{cellspace.tmp}%
  \endgroup
  \nobreak \@nobreaktrue \vskip-\baselineskip \vskip\z@skip
  \begingroup \@nameuse{\@showexampleenv}%
    \small
    \input{cellspace.tmp}%
  \@nameuse{end\@showexampleenv}\endgroup
  \par \bigbreak
  \@endparenv
}
\newcommand*{\@showexample}[1][center]{%
  \egroup
  \def \@showexampleenv {#1}%
  \begingroup
    \verbatimwrite{cellspace.tmp}%
}
\makeatother

% -----------------------------------------------------------------

\setlength{\cellspacetoplimit}{2pt}
\setlength{\cellspacebottomlimit}{2pt}
\setcounter{secnumdepth}{0}
\addparagraphcolumntypes{X}

% -----------------------------------------------------------------

\GetFileInfo{cellspace.sty}

\title  {The \pack{cellspace} package}
\author {Josselin Noirel\footnote{This document corresponds to version
  \fileversion\ (\filedate) of \pack{\filename}.}\\%
\url{https://github.com/JosselinNoirel/cellspace}}
\date   {\filedate---\fileversion}

\begin{document}

\maketitle

\begin{abstract}
  This package is intended to allow automatic spacing out of the lines of
  an array.  People often complain about text touching the \cmd{hline} of
  a tabular when it is too high or too deep.  For example (with the
  \pack{amsmath} package) with \tex{\textbackslash dfrac\{1\}\{2\}}
  surrounded by \cmd{hline}s.  This package provides a modifier \tex{S} to
  usual column types (\tex{l}, \tex{c}, \tex{r}, \tex{p}, \tex{m},
  and~\tex{b}) that ensures a minimal spacing between rules and cells of
  an array.
\end{abstract}

\section{Introduction}

The mechanism used by \LaTeX{} to build tables---using struts---has an
important consequence: the cells of a table that extend too much tend to
touch horizontal rules.  First example:
%
\begin{showexample}
\begin{tabular}{cc}
  \hline
  \itshape Bond &
  \itshape Distance ($\mathrm{\mathring{A}})$ \\
  \hline
  C--C          & $1.53$ \\
  C--H          & $1.10$ \\
  \hline
\end{tabular}
\end{showexample}
%
Second example:
%
\begin{showexample}
\begin{tabular}{c}
  \hline
  \itshape Formula \\
  \hline
  $\displaystyle
   e = 1 + \frac{1}{2} + \frac{1}{6}  + \cdots
                       + \frac{1}{k!} + \cdots $ \\
  $\displaystyle
   \pi = 4 \left( 1 - \frac{1}{3} + \frac{1}{5} + \cdots
                    + \frac{(-1)^k}{2 k + 1}
                    + \cdots \right)$ \\
  \hline
\end{tabular}
\end{showexample}
%
The classical workarounds \cmd{\textbackslash}\oarg{dimen} and
\cmd{noalign}\tex{\{}\cmd{vspace}\marg{dimen}\tex{\}} are not very powerful
and need fine tuning.

\section{How to solve this?}

The \pack{cellspace} loads several packages to carry out its job:
\pack{array}, \pack{ifthen}, and~\pack{calc} (along \tex{amsmath} when
the option \tex{math} is passed to \tex{cellspace}).  It redefines several
internals, still not too many (\cmd{@startpbox} and \cmd{@endpbox}).  By
default, the tables will behave as usual.  To improve the spacing of your
tables, you must change the table preamble and prepend \tex{S} to the
column types \tex{l}, \tex{c}, \tex{r}.  The same holds for the paragraph
columns \tex{p}, \tex{m}, and~\tex{b}, except that they must be surrounded
by an extra pair of braces.  For instance, the default behaviour of a
table beginning with
%
\begin{verbatim}
\begin{tabular}{l l l p{3cm}}
\end{verbatim}
%
should be changed into
%
\begin{verbatim}
\begin{tabular}{Sl Sl Sl S{p{3cm}}}
\end{verbatim}

The \pack{cellspace} has two parameters governing the spacing of the
cells.  The dimension \cmd{cellspacetoplimit} is the minimal spacing
required between the actual text the cell is made up of and the top of the
cell (where \cmd{hline}s may appear), if the spacing is less than this
threshold, a space of \cmd{cellspacetoplimit} is added, otherwise nothing
is done. Conversely, at the bottom the dimension
\cmd{cellspacebottomlimit} is the minimal space required between the
bottom of the cell and the text itself. This parameters can be changed in
the document preamble using \cmd{setlength}.  Together with
\pack{booktabs}, the results look rather good.  \pack{cellspace} may work
with other column types (defined through \cmd{newcolumntype} for
instance); it will assume by default that an unknown column is a LR-mode
column.  Otherwise you have to tell \pack{cellspace} something like
%
\begin{verbatim}
\addparagraphcolumntypes{X}
\end{verbatim}
%
(The previous example is needed to make \pack{tabularx} work with
\pack{cellspace}.)

The following examples use a setting of $2\,\mathrm{pt}$ for both top and
bottom space limits.
%
\compare{%
  \begin{tabular}[t]{cc}
    \hline
    \itshape Bond &
    \itshape Distance ($\mathrm{\mathring{A}})$ \\
    \hline
    C--C          & $1.53$ \\
    C--H          & $1.10$ \\
    \hline
  \end{tabular}
}{%
  \begin{tabular}[t]{Sc Sc}
    \hline
    \itshape Bond &
    \itshape Distance ($\mathrm{\mathring{A}})$ \\
    \hline
    C--C          & $1.53$ \\
    C--H          & $1.10$ \\
    \hline
  \end{tabular}
}
%
Some space is added at the top the first row because it doesn't fulfil
the condition imposed by the package.  Similarly some space is added at
the bottom of the first row.  It has to be noted, however, that no space
is added in other rows because they are far enough from the cell
separations.

\compare{%
  \begin{tabular}[t]{c}
    \hline
    \itshape Formula \\
    \hline
    $\displaystyle
    e = 1 + \frac{1}{2} + \frac{1}{6}  + \cdots
                         + \frac{1}{k!} + \cdots $ \\
    $\displaystyle
     \pi = 4 \left( 1 - \frac{1}{3} + \frac{1}{5} + \cdots
                      + \frac{(-1)^k}{2 k + 1}
                      + \cdots \right)$ \\
    \hline
  \end{tabular}
}{%
  \begin{tabular}[t]{Sc}
    \hline
    \itshape Formula \\
    \hline
    $\displaystyle
    e = 1 + \frac{1}{2} + \frac{1}{6}  + \cdots
                         + \frac{1}{k!} + \cdots $ \\
    $\displaystyle
     \pi = 4 \left( 1 - \frac{1}{3} + \frac{1}{5} + \cdots
                      + \frac{(-1)^k}{2 k + 1}
                      + \cdots \right)$ \\
    \hline
  \end{tabular}
}
%
Here it can be immediately remarked that the `Formula' row has the same
height in both examples.  Only the formula lines are affected.  The best
layout is obtained with \pack{booktabs}:
%
\compare{%
  \begin{tabular}[t]{@{} c @{}}
    \toprule
    \itshape Formula \\
    \midrule
    $\displaystyle
    e = 1 + \frac{1}{2} + \frac{1}{6}  + \cdots
                         + \frac{1}{k!} + \cdots $ \\
    $\displaystyle
     \pi = 4 \left( 1 - \frac{1}{3} + \frac{1}{5} + \cdots
                      + \frac{(-1)^k}{2 k + 1}
                      + \cdots \right)$ \\
    \bottomrule
  \end{tabular}
}{%
  \begin{tabular}[t]{@{} Sc @{}}
    \toprule
    \itshape Formula \\
    \midrule
    $\displaystyle
    e = 1 + \frac{1}{2} + \frac{1}{6}  + \cdots
                         + \frac{1}{k!} + \cdots $ \\
    $\displaystyle
     \pi = 4 \left( 1 - \frac{1}{3} + \frac{1}{5} + \cdots
                      + \frac{(-1)^k}{2 k + 1}
                      + \cdots \right)$ \\
    \bottomrule
  \end{tabular}
}

The following example shows that how it works with paragraph columns.
%
\compare{%
  \begin{tabular}[t]{p{5cm}}
    \hline
    $\displaystyle \frac{1}{2}$
    A A A A A A A A A A A A A A
    A A A A A A A A A A A A A A
    A A A A A A A A A A A A A A
    A A A A A A A A A A A A A A
    A A A A A A A A A A A A A A
    $\displaystyle \frac{1}{2}$ \\
    \hline
  \end{tabular}
}{%
  \begin{tabular}[t]{S{p{5cm}}}
    \hline
    $\displaystyle \frac{1}{2}$
    A A A A A A A A A A A A A A
    A A A A A A A A A A A A A A
    A A A A A A A A A A A A A A
    A A A A A A A A A A A A A A
    A A A A A A A A A A A A A A
    $\displaystyle \frac{1}{2}$ \\
    \hline
  \end{tabular}
}

\section{Matrices}

The package can be loaded with the options \tex{nomath} (default) and
\tex{math}.  The latter does two things: it first loads the
\tex{amsmath} package and then redefines
the command that is invoked when one typesets matrices
(\tex{\textbackslash env@matrix} --- this improvement was suggested by
Bastien Roucaries).  The following
examples illustrate what happens with (left) or without (right)
\pack{cellspace}'s correction in a \tex{pmatrix} environment.
%
\[
\begingroup
\makeatletter
  \def\env@matrix{\hskip -\arraycolsep
  \let\@ifnextchar\new@ifnextchar
  \array{*\c@MaxMatrixCols c}}
\makeatother
%
\text{\tex{nomath} option\ }%
\begin{pmatrix}
+\frac{1}{2} & +\frac{1}{3} \\
-\frac{1}{3} & +\frac{1}{2} \\
\end{pmatrix}
\endgroup
\qquad
\text{\tex{math} option\ }%
\begin{pmatrix}
+\frac{1}{2} & +\frac{1}{3} \\
-\frac{1}{3} & +\frac{1}{2} \\
\end{pmatrix}
\]

As a~consequence, the \tex{amsmath} package can be loaded beforehand
with other packages (such as \tex{empheq} or \tex{mathtools}), were an
incompatibility to arise from one's loading it later.

\section{Using a different column name}

An other column name than \tex{S} can be used if \tex{S} conflicts
with the user's own definitions or a package's.  A keyval option can
be passed to the package to that effect;
%
\begin{verbatim}
\usepackage[column=O]{cellspace}
\end{verbatim}
%
can be used to provide cellspace's functionality through the \tex{O}
column modifier instead of the \tex{S} column modifier.


\section{Bugs and limitations}

This package hasn't been heavily tested, so there may be plenty of bugs.
As usual, bugs will certainly arise in complicated situations because
tables can become very complicated (for instance, this package hasn't been
designed with nested tables into mind).  In simple cases however, it should
work nicely.  The package loads \pack{array} ensuring a peaceful
collaboration with each other.  Other packages of special interest haven't
been tested: \pack{tabularx}\footnote{A one minute test seems to indicate
  that it works provided \pack{tabularx} is loaded after \pack{cellspace}.
  The example on this page illustrates this.  A thirty second test
indicates that \pack{cellspace} works correctly with \pack{longtable} as
long as only LR-mode columns are concerned.} and \pack{longtable} in
particular.
%
\begin{trivlist}\item
\begin{tabularx}{\linewidth}{l SX}
  \hline
  A       & B B B B B B B B \\
  A A     & B B B B B B B B
            B B B B B B B B \\
  A A A A & $\dfrac{1}{2}$B B B B B B B B
            B B B B B B B B
            B B B B B B B B
            B B B B B B B B$\dfrac{1}{2}$ \\
  \hline
\end{tabularx}
\end{trivlist}

\pack{tabls}\footnote{Thanks to Jim Hefferon for pointing this out.} is a
package that does a similar job as \pack{cellspace}.
The differences are: \pack{tabls} affects a whole table while
\pack{cellspace} affects only the columns on which applies the modifier
\tex{S}.  Each cell affected by \pack{cellspace} is affected in the same
way: there must be enough space above ($>\cmd{cellspacetoplimit}$) and
below the text ($>\cmd{cellspacebottomlimit}$), even if no \cmd{hline} is
present.  In the other hand, \pack{tabls} ensures a minimal distance using
one threshold parameter \cmd{tablinesep} applying between rows not
separated by a rule and using a systematic spacer \cmd{arraylinesep} when
there is a rule (whereas \pack{cellspace} can leave a cell as is if the
distance to the rule is big enough).  As a last point, \pack{tabls}, as
far as I can see, cannot work with \pack{array}.

\section{What's new}

\begin{description}
\item [v1.8.1] Package becomes aware of the latest version of the
      array package
\end{description}

\end{document}
%
  \@nameuse{end\@showexampleenv}\endgroup
  \par \bigbreak
  \@endparenv
}
\newcommand*{\@showexample}[1][center]{%
  \egroup
  \def \@showexampleenv {#1}%
  \begingroup
    \verbatimwrite{cellspace.tmp}%
}
\makeatother

% -----------------------------------------------------------------

\setlength{\cellspacetoplimit}{2pt}
\setlength{\cellspacebottomlimit}{2pt}
\setcounter{secnumdepth}{0}
\addparagraphcolumntypes{X}

% -----------------------------------------------------------------

\GetFileInfo{cellspace.sty}

\title  {The \pack{cellspace} package}
\author {Josselin Noirel\footnote{This document corresponds to version
  \fileversion\ (\filedate) of \pack{\filename}.}\\%
\url{https://github.com/JosselinNoirel/cellspace}}
\date   {\filedate---\fileversion}

\begin{document}

\maketitle

\begin{abstract}
  This package is intended to allow automatic spacing out of the lines of
  an array.  People often complain about text touching the \cmd{hline} of
  a tabular when it is too high or too deep.  For example (with the
  \pack{amsmath} package) with \tex{\textbackslash dfrac\{1\}\{2\}}
  surrounded by \cmd{hline}s.  This package provides a modifier \tex{S} to
  usual column types (\tex{l}, \tex{c}, \tex{r}, \tex{p}, \tex{m},
  and~\tex{b}) that ensures a minimal spacing between rules and cells of
  an array.
\end{abstract}

\section{Introduction}

The mechanism used by \LaTeX{} to build tables---using struts---has an
important consequence: the cells of a table that extend too much tend to
touch horizontal rules.  First example:
%
\begin{showexample}
\begin{tabular}{cc}
  \hline
  \itshape Bond &
  \itshape Distance ($\mathrm{\mathring{A}})$ \\
  \hline
  C--C          & $1.53$ \\
  C--H          & $1.10$ \\
  \hline
\end{tabular}
\end{showexample}
%
Second example:
%
\begin{showexample}
\begin{tabular}{c}
  \hline
  \itshape Formula \\
  \hline
  $\displaystyle
   e = 1 + \frac{1}{2} + \frac{1}{6}  + \cdots
                       + \frac{1}{k!} + \cdots $ \\
  $\displaystyle
   \pi = 4 \left( 1 - \frac{1}{3} + \frac{1}{5} + \cdots
                    + \frac{(-1)^k}{2 k + 1}
                    + \cdots \right)$ \\
  \hline
\end{tabular}
\end{showexample}
%
The classical workarounds \cmd{\textbackslash}\oarg{dimen} and
\cmd{noalign}\tex{\{}\cmd{vspace}\marg{dimen}\tex{\}} are not very powerful
and need fine tuning.

\section{How to solve this?}

The \pack{cellspace} loads several packages to carry out its job:
\pack{array}, \pack{ifthen}, and~\pack{calc} (along \tex{amsmath} when
the option \tex{math} is passed to \tex{cellspace}).  It redefines several
internals, still not too many (\cmd{@startpbox} and \cmd{@endpbox}).  By
default, the tables will behave as usual.  To improve the spacing of your
tables, you must change the table preamble and prepend \tex{S} to the
column types \tex{l}, \tex{c}, \tex{r}.  The same holds for the paragraph
columns \tex{p}, \tex{m}, and~\tex{b}, except that they must be surrounded
by an extra pair of braces.  For instance, the default behaviour of a
table beginning with
%
\begin{verbatim}
\begin{tabular}{l l l p{3cm}}
\end{verbatim}
%
should be changed into
%
\begin{verbatim}
\begin{tabular}{Sl Sl Sl S{p{3cm}}}
\end{verbatim}

The \pack{cellspace} has two parameters governing the spacing of the
cells.  The dimension \cmd{cellspacetoplimit} is the minimal spacing
required between the actual text the cell is made up of and the top of the
cell (where \cmd{hline}s may appear), if the spacing is less than this
threshold, a space of \cmd{cellspacetoplimit} is added, otherwise nothing
is done. Conversely, at the bottom the dimension
\cmd{cellspacebottomlimit} is the minimal space required between the
bottom of the cell and the text itself. This parameters can be changed in
the document preamble using \cmd{setlength}.  Together with
\pack{booktabs}, the results look rather good.  \pack{cellspace} may work
with other column types (defined through \cmd{newcolumntype} for
instance); it will assume by default that an unknown column is a LR-mode
column.  Otherwise you have to tell \pack{cellspace} something like
%
\begin{verbatim}
\addparagraphcolumntypes{X}
\end{verbatim}
%
(The previous example is needed to make \pack{tabularx} work with
\pack{cellspace}.)

The following examples use a setting of $2\,\mathrm{pt}$ for both top and
bottom space limits.
%
\compare{%
  \begin{tabular}[t]{cc}
    \hline
    \itshape Bond &
    \itshape Distance ($\mathrm{\mathring{A}})$ \\
    \hline
    C--C          & $1.53$ \\
    C--H          & $1.10$ \\
    \hline
  \end{tabular}
}{%
  \begin{tabular}[t]{Sc Sc}
    \hline
    \itshape Bond &
    \itshape Distance ($\mathrm{\mathring{A}})$ \\
    \hline
    C--C          & $1.53$ \\
    C--H          & $1.10$ \\
    \hline
  \end{tabular}
}
%
Some space is added at the top the first row because it doesn't fulfil
the condition imposed by the package.  Similarly some space is added at
the bottom of the first row.  It has to be noted, however, that no space
is added in other rows because they are far enough from the cell
separations.

\compare{%
  \begin{tabular}[t]{c}
    \hline
    \itshape Formula \\
    \hline
    $\displaystyle
    e = 1 + \frac{1}{2} + \frac{1}{6}  + \cdots
                         + \frac{1}{k!} + \cdots $ \\
    $\displaystyle
     \pi = 4 \left( 1 - \frac{1}{3} + \frac{1}{5} + \cdots
                      + \frac{(-1)^k}{2 k + 1}
                      + \cdots \right)$ \\
    \hline
  \end{tabular}
}{%
  \begin{tabular}[t]{Sc}
    \hline
    \itshape Formula \\
    \hline
    $\displaystyle
    e = 1 + \frac{1}{2} + \frac{1}{6}  + \cdots
                         + \frac{1}{k!} + \cdots $ \\
    $\displaystyle
     \pi = 4 \left( 1 - \frac{1}{3} + \frac{1}{5} + \cdots
                      + \frac{(-1)^k}{2 k + 1}
                      + \cdots \right)$ \\
    \hline
  \end{tabular}
}
%
Here it can be immediately remarked that the `Formula' row has the same
height in both examples.  Only the formula lines are affected.  The best
layout is obtained with \pack{booktabs}:
%
\compare{%
  \begin{tabular}[t]{@{} c @{}}
    \toprule
    \itshape Formula \\
    \midrule
    $\displaystyle
    e = 1 + \frac{1}{2} + \frac{1}{6}  + \cdots
                         + \frac{1}{k!} + \cdots $ \\
    $\displaystyle
     \pi = 4 \left( 1 - \frac{1}{3} + \frac{1}{5} + \cdots
                      + \frac{(-1)^k}{2 k + 1}
                      + \cdots \right)$ \\
    \bottomrule
  \end{tabular}
}{%
  \begin{tabular}[t]{@{} Sc @{}}
    \toprule
    \itshape Formula \\
    \midrule
    $\displaystyle
    e = 1 + \frac{1}{2} + \frac{1}{6}  + \cdots
                         + \frac{1}{k!} + \cdots $ \\
    $\displaystyle
     \pi = 4 \left( 1 - \frac{1}{3} + \frac{1}{5} + \cdots
                      + \frac{(-1)^k}{2 k + 1}
                      + \cdots \right)$ \\
    \bottomrule
  \end{tabular}
}

The following example shows that how it works with paragraph columns.
%
\compare{%
  \begin{tabular}[t]{p{5cm}}
    \hline
    $\displaystyle \frac{1}{2}$
    A A A A A A A A A A A A A A
    A A A A A A A A A A A A A A
    A A A A A A A A A A A A A A
    A A A A A A A A A A A A A A
    A A A A A A A A A A A A A A
    $\displaystyle \frac{1}{2}$ \\
    \hline
  \end{tabular}
}{%
  \begin{tabular}[t]{S{p{5cm}}}
    \hline
    $\displaystyle \frac{1}{2}$
    A A A A A A A A A A A A A A
    A A A A A A A A A A A A A A
    A A A A A A A A A A A A A A
    A A A A A A A A A A A A A A
    A A A A A A A A A A A A A A
    $\displaystyle \frac{1}{2}$ \\
    \hline
  \end{tabular}
}

\section{Matrices}

The package can be loaded with the options \tex{nomath} (default) and
\tex{math}.  The latter does two things: it first loads the
\tex{amsmath} package and then redefines
the command that is invoked when one typesets matrices
(\tex{\textbackslash env@matrix} --- this improvement was suggested by
Bastien Roucaries).  The following
examples illustrate what happens with (left) or without (right)
\pack{cellspace}'s correction in a \tex{pmatrix} environment.
%
\[
\begingroup
\makeatletter
  \def\env@matrix{\hskip -\arraycolsep
  \let\@ifnextchar\new@ifnextchar
  \array{*\c@MaxMatrixCols c}}
\makeatother
%
\text{\tex{nomath} option\ }%
\begin{pmatrix}
+\frac{1}{2} & +\frac{1}{3} \\
-\frac{1}{3} & +\frac{1}{2} \\
\end{pmatrix}
\endgroup
\qquad
\text{\tex{math} option\ }%
\begin{pmatrix}
+\frac{1}{2} & +\frac{1}{3} \\
-\frac{1}{3} & +\frac{1}{2} \\
\end{pmatrix}
\]

As a~consequence, the \tex{amsmath} package can be loaded beforehand
with other packages (such as \tex{empheq} or \tex{mathtools}), were an
incompatibility to arise from one's loading it later.

\section{Using a different column name}

An other column name than \tex{S} can be used if \tex{S} conflicts
with the user's own definitions or a package's.  A keyval option can
be passed to the package to that effect;
%
\begin{verbatim}
\usepackage[column=O]{cellspace}
\end{verbatim}
%
can be used to provide cellspace's functionality through the \tex{O}
column modifier instead of the \tex{S} column modifier.


\section{Bugs and limitations}

This package hasn't been heavily tested, so there may be plenty of bugs.
As usual, bugs will certainly arise in complicated situations because
tables can become very complicated (for instance, this package hasn't been
designed with nested tables into mind).  In simple cases however, it should
work nicely.  The package loads \pack{array} ensuring a peaceful
collaboration with each other.  Other packages of special interest haven't
been tested: \pack{tabularx}\footnote{A one minute test seems to indicate
  that it works provided \pack{tabularx} is loaded after \pack{cellspace}.
  The example on this page illustrates this.  A thirty second test
indicates that \pack{cellspace} works correctly with \pack{longtable} as
long as only LR-mode columns are concerned.} and \pack{longtable} in
particular.
%
\begin{trivlist}\item
\begin{tabularx}{\linewidth}{l SX}
  \hline
  A       & B B B B B B B B \\
  A A     & B B B B B B B B
            B B B B B B B B \\
  A A A A & $\dfrac{1}{2}$B B B B B B B B
            B B B B B B B B
            B B B B B B B B
            B B B B B B B B$\dfrac{1}{2}$ \\
  \hline
\end{tabularx}
\end{trivlist}

\pack{tabls}\footnote{Thanks to Jim Hefferon for pointing this out.} is a
package that does a similar job as \pack{cellspace}.
The differences are: \pack{tabls} affects a whole table while
\pack{cellspace} affects only the columns on which applies the modifier
\tex{S}.  Each cell affected by \pack{cellspace} is affected in the same
way: there must be enough space above ($>\cmd{cellspacetoplimit}$) and
below the text ($>\cmd{cellspacebottomlimit}$), even if no \cmd{hline} is
present.  In the other hand, \pack{tabls} ensures a minimal distance using
one threshold parameter \cmd{tablinesep} applying between rows not
separated by a rule and using a systematic spacer \cmd{arraylinesep} when
there is a rule (whereas \pack{cellspace} can leave a cell as is if the
distance to the rule is big enough).  As a last point, \pack{tabls}, as
far as I can see, cannot work with \pack{array}.

\section{What's new}

\begin{description}
\item [v1.8.1] Package becomes aware of the latest version of the
      array package
\end{description}

\end{document}
%
  \@nameuse{end\@showexampleenv}\endgroup
  \par \bigbreak
  \@endparenv
}
\newcommand*{\@showexample}[1][center]{%
  \egroup
  \def \@showexampleenv {#1}%
  \begingroup
    \verbatimwrite{cellspace.tmp}%
}
\makeatother

% -----------------------------------------------------------------

\setlength{\cellspacetoplimit}{2pt}
\setlength{\cellspacebottomlimit}{2pt}
\setcounter{secnumdepth}{0}
\addparagraphcolumntypes{X}

% -----------------------------------------------------------------

\GetFileInfo{cellspace.sty}

\title  {The \pack{cellspace} package}
\author {Josselin Noirel\footnote{This document corresponds to version
  \fileversion\ (\filedate) of \pack{\filename}.}\\%
\url{https://github.com/JosselinNoirel/cellspace}}
\date   {\filedate---\fileversion}

\begin{document}

\maketitle

\begin{abstract}
  This package is intended to allow automatic spacing out of the lines of
  an array.  People often complain about text touching the \cmd{hline} of
  a tabular when it is too high or too deep.  For example (with the
  \pack{amsmath} package) with \tex{\textbackslash dfrac\{1\}\{2\}}
  surrounded by \cmd{hline}s.  This package provides a modifier \tex{S} to
  usual column types (\tex{l}, \tex{c}, \tex{r}, \tex{p}, \tex{m},
  and~\tex{b}) that ensures a minimal spacing between rules and cells of
  an array.
\end{abstract}

\section{Introduction}

The mechanism used by \LaTeX{} to build tables---using struts---has an
important consequence: the cells of a table that extend too much tend to
touch horizontal rules.  First example:
%
\begin{showexample}
\begin{tabular}{cc}
  \hline
  \itshape Bond &
  \itshape Distance ($\mathrm{\mathring{A}})$ \\
  \hline
  C--C          & $1.53$ \\
  C--H          & $1.10$ \\
  \hline
\end{tabular}
\end{showexample}
%
Second example:
%
\begin{showexample}
\begin{tabular}{c}
  \hline
  \itshape Formula \\
  \hline
  $\displaystyle
   e = 1 + \frac{1}{2} + \frac{1}{6}  + \cdots
                       + \frac{1}{k!} + \cdots $ \\
  $\displaystyle
   \pi = 4 \left( 1 - \frac{1}{3} + \frac{1}{5} + \cdots
                    + \frac{(-1)^k}{2 k + 1}
                    + \cdots \right)$ \\
  \hline
\end{tabular}
\end{showexample}
%
The classical workarounds \cmd{\textbackslash}\oarg{dimen} and
\cmd{noalign}\tex{\{}\cmd{vspace}\marg{dimen}\tex{\}} are not very powerful
and need fine tuning.

\section{How to solve this?}

The \pack{cellspace} loads several packages to carry out its job:
\pack{array}, \pack{ifthen}, and~\pack{calc} (along \tex{amsmath} when
the option \tex{math} is passed to \tex{cellspace}).  It redefines several
internals, still not too many (\cmd{@startpbox} and \cmd{@endpbox}).  By
default, the tables will behave as usual.  To improve the spacing of your
tables, you must change the table preamble and prepend \tex{S} to the
column types \tex{l}, \tex{c}, \tex{r}.  The same holds for the paragraph
columns \tex{p}, \tex{m}, and~\tex{b}, except that they must be surrounded
by an extra pair of braces.  For instance, the default behaviour of a
table beginning with
%
\begin{verbatim}
\begin{tabular}{l l l p{3cm}}
\end{verbatim}
%
should be changed into
%
\begin{verbatim}
\begin{tabular}{Sl Sl Sl S{p{3cm}}}
\end{verbatim}

The \pack{cellspace} has two parameters governing the spacing of the
cells.  The dimension \cmd{cellspacetoplimit} is the minimal spacing
required between the actual text the cell is made up of and the top of the
cell (where \cmd{hline}s may appear), if the spacing is less than this
threshold, a space of \cmd{cellspacetoplimit} is added, otherwise nothing
is done. Conversely, at the bottom the dimension
\cmd{cellspacebottomlimit} is the minimal space required between the
bottom of the cell and the text itself. This parameters can be changed in
the document preamble using \cmd{setlength}.  Together with
\pack{booktabs}, the results look rather good.  \pack{cellspace} may work
with other column types (defined through \cmd{newcolumntype} for
instance); it will assume by default that an unknown column is a LR-mode
column.  Otherwise you have to tell \pack{cellspace} something like
%
\begin{verbatim}
\addparagraphcolumntypes{X}
\end{verbatim}
%
(The previous example is needed to make \pack{tabularx} work with
\pack{cellspace}.)

The following examples use a setting of $2\,\mathrm{pt}$ for both top and
bottom space limits.
%
\compare{%
  \begin{tabular}[t]{cc}
    \hline
    \itshape Bond &
    \itshape Distance ($\mathrm{\mathring{A}})$ \\
    \hline
    C--C          & $1.53$ \\
    C--H          & $1.10$ \\
    \hline
  \end{tabular}
}{%
  \begin{tabular}[t]{Sc Sc}
    \hline
    \itshape Bond &
    \itshape Distance ($\mathrm{\mathring{A}})$ \\
    \hline
    C--C          & $1.53$ \\
    C--H          & $1.10$ \\
    \hline
  \end{tabular}
}
%
Some space is added at the top the first row because it doesn't fulfil
the condition imposed by the package.  Similarly some space is added at
the bottom of the first row.  It has to be noted, however, that no space
is added in other rows because they are far enough from the cell
separations.

\compare{%
  \begin{tabular}[t]{c}
    \hline
    \itshape Formula \\
    \hline
    $\displaystyle
    e = 1 + \frac{1}{2} + \frac{1}{6}  + \cdots
                         + \frac{1}{k!} + \cdots $ \\
    $\displaystyle
     \pi = 4 \left( 1 - \frac{1}{3} + \frac{1}{5} + \cdots
                      + \frac{(-1)^k}{2 k + 1}
                      + \cdots \right)$ \\
    \hline
  \end{tabular}
}{%
  \begin{tabular}[t]{Sc}
    \hline
    \itshape Formula \\
    \hline
    $\displaystyle
    e = 1 + \frac{1}{2} + \frac{1}{6}  + \cdots
                         + \frac{1}{k!} + \cdots $ \\
    $\displaystyle
     \pi = 4 \left( 1 - \frac{1}{3} + \frac{1}{5} + \cdots
                      + \frac{(-1)^k}{2 k + 1}
                      + \cdots \right)$ \\
    \hline
  \end{tabular}
}
%
Here it can be immediately remarked that the `Formula' row has the same
height in both examples.  Only the formula lines are affected.  The best
layout is obtained with \pack{booktabs}:
%
\compare{%
  \begin{tabular}[t]{@{} c @{}}
    \toprule
    \itshape Formula \\
    \midrule
    $\displaystyle
    e = 1 + \frac{1}{2} + \frac{1}{6}  + \cdots
                         + \frac{1}{k!} + \cdots $ \\
    $\displaystyle
     \pi = 4 \left( 1 - \frac{1}{3} + \frac{1}{5} + \cdots
                      + \frac{(-1)^k}{2 k + 1}
                      + \cdots \right)$ \\
    \bottomrule
  \end{tabular}
}{%
  \begin{tabular}[t]{@{} Sc @{}}
    \toprule
    \itshape Formula \\
    \midrule
    $\displaystyle
    e = 1 + \frac{1}{2} + \frac{1}{6}  + \cdots
                         + \frac{1}{k!} + \cdots $ \\
    $\displaystyle
     \pi = 4 \left( 1 - \frac{1}{3} + \frac{1}{5} + \cdots
                      + \frac{(-1)^k}{2 k + 1}
                      + \cdots \right)$ \\
    \bottomrule
  \end{tabular}
}

The following example shows that how it works with paragraph columns.
%
\compare{%
  \begin{tabular}[t]{p{5cm}}
    \hline
    $\displaystyle \frac{1}{2}$
    A A A A A A A A A A A A A A
    A A A A A A A A A A A A A A
    A A A A A A A A A A A A A A
    A A A A A A A A A A A A A A
    A A A A A A A A A A A A A A
    $\displaystyle \frac{1}{2}$ \\
    \hline
  \end{tabular}
}{%
  \begin{tabular}[t]{S{p{5cm}}}
    \hline
    $\displaystyle \frac{1}{2}$
    A A A A A A A A A A A A A A
    A A A A A A A A A A A A A A
    A A A A A A A A A A A A A A
    A A A A A A A A A A A A A A
    A A A A A A A A A A A A A A
    $\displaystyle \frac{1}{2}$ \\
    \hline
  \end{tabular}
}

\section{Matrices}

The package can be loaded with the options \tex{nomath} (default) and
\tex{math}.  The latter does two things: it first loads the
\tex{amsmath} package and then redefines
the command that is invoked when one typesets matrices
(\tex{\textbackslash env@matrix} --- this improvement was suggested by
Bastien Roucaries).  The following
examples illustrate what happens with (left) or without (right)
\pack{cellspace}'s correction in a \tex{pmatrix} environment.
%
\[
\begingroup
\makeatletter
  \def\env@matrix{\hskip -\arraycolsep
  \let\@ifnextchar\new@ifnextchar
  \array{*\c@MaxMatrixCols c}}
\makeatother
%
\text{\tex{nomath} option\ }%
\begin{pmatrix}
+\frac{1}{2} & +\frac{1}{3} \\
-\frac{1}{3} & +\frac{1}{2} \\
\end{pmatrix}
\endgroup
\qquad
\text{\tex{math} option\ }%
\begin{pmatrix}
+\frac{1}{2} & +\frac{1}{3} \\
-\frac{1}{3} & +\frac{1}{2} \\
\end{pmatrix}
\]

As a~consequence, the \tex{amsmath} package can be loaded beforehand
with other packages (such as \tex{empheq} or \tex{mathtools}), were an
incompatibility to arise from one's loading it later.

\section{Using a different column name}

An other column name than \tex{S} can be used if \tex{S} conflicts
with the user's own definitions or a package's.  A keyval option can
be passed to the package to that effect;
%
\begin{verbatim}
\usepackage[column=O]{cellspace}
\end{verbatim}
%
can be used to provide cellspace's functionality through the \tex{O}
column modifier instead of the \tex{S} column modifier.


\section{Bugs and limitations}

This package hasn't been heavily tested, so there may be plenty of bugs.
As usual, bugs will certainly arise in complicated situations because
tables can become very complicated (for instance, this package hasn't been
designed with nested tables into mind).  In simple cases however, it should
work nicely.  The package loads \pack{array} ensuring a peaceful
collaboration with each other.  Other packages of special interest haven't
been tested: \pack{tabularx}\footnote{A one minute test seems to indicate
  that it works provided \pack{tabularx} is loaded after \pack{cellspace}.
  The example on this page illustrates this.  A thirty second test
indicates that \pack{cellspace} works correctly with \pack{longtable} as
long as only LR-mode columns are concerned.} and \pack{longtable} in
particular.
%
\begin{trivlist}\item
\begin{tabularx}{\linewidth}{l SX}
  \hline
  A       & B B B B B B B B \\
  A A     & B B B B B B B B
            B B B B B B B B \\
  A A A A & $\dfrac{1}{2}$B B B B B B B B
            B B B B B B B B
            B B B B B B B B
            B B B B B B B B$\dfrac{1}{2}$ \\
  \hline
\end{tabularx}
\end{trivlist}

\pack{tabls}\footnote{Thanks to Jim Hefferon for pointing this out.} is a
package that does a similar job as \pack{cellspace}.
The differences are: \pack{tabls} affects a whole table while
\pack{cellspace} affects only the columns on which applies the modifier
\tex{S}.  Each cell affected by \pack{cellspace} is affected in the same
way: there must be enough space above ($>\cmd{cellspacetoplimit}$) and
below the text ($>\cmd{cellspacebottomlimit}$), even if no \cmd{hline} is
present.  In the other hand, \pack{tabls} ensures a minimal distance using
one threshold parameter \cmd{tablinesep} applying between rows not
separated by a rule and using a systematic spacer \cmd{arraylinesep} when
there is a rule (whereas \pack{cellspace} can leave a cell as is if the
distance to the rule is big enough).  As a last point, \pack{tabls}, as
far as I can see, cannot work with \pack{array}.

\section{What's new}

\begin{description}
\item [v1.8.1] Package becomes aware of the latest version of the
      array package
\end{description}

\end{document}
%
  \@nameuse{end\@showexampleenv}\endgroup
  \par \bigbreak
  \@endparenv
}
\newcommand*{\@showexample}[1][center]{%
  \egroup
  \def \@showexampleenv {#1}%
  \begingroup
    \verbatimwrite{cellspace.tmp}%
}
\makeatother

% -----------------------------------------------------------------

\setlength{\cellspacetoplimit}{2pt}
\setlength{\cellspacebottomlimit}{2pt}
\setcounter{secnumdepth}{0}
\addparagraphcolumntypes{X}

% -----------------------------------------------------------------

\GetFileInfo{cellspace.sty}

\title  {The \pack{cellspace} package}
\author {Josselin Noirel\footnote{This document corresponds to version
  \fileversion\ (\filedate) of \pack{\filename}.}\\%
\url{https://github.com/JosselinNoirel/cellspace}}
\date   {\filedate---\fileversion}

\begin{document}

\maketitle

\begin{abstract}
  This package is intended to allow automatic spacing out of the lines of
  an array.  People often complain about text touching the \cmd{hline} of
  a tabular when it is too high or too deep.  For example (with the
  \pack{amsmath} package) with \tex{\textbackslash dfrac\{1\}\{2\}}
  surrounded by \cmd{hline}s.  This package provides a modifier \tex{S} to
  usual column types (\tex{l}, \tex{c}, \tex{r}, \tex{p}, \tex{m},
  and~\tex{b}) that ensures a minimal spacing between rules and cells of
  an array.
\end{abstract}

\section{Introduction}

The mechanism used by \LaTeX{} to build tables---using struts---has an
important consequence: the cells of a table that extend too much tend to
touch horizontal rules.  First example:
%
\begin{showexample}
\begin{tabular}{cc}
  \hline
  \itshape Bond &
  \itshape Distance ($\mathrm{\mathring{A}})$ \\
  \hline
  C--C          & $1.53$ \\
  C--H          & $1.10$ \\
  \hline
\end{tabular}
\end{showexample}
%
Second example:
%
\begin{showexample}
\begin{tabular}{c}
  \hline
  \itshape Formula \\
  \hline
  $\displaystyle
   e = 1 + \frac{1}{2} + \frac{1}{6}  + \cdots
                       + \frac{1}{k!} + \cdots $ \\
  $\displaystyle
   \pi = 4 \left( 1 - \frac{1}{3} + \frac{1}{5} + \cdots
                    + \frac{(-1)^k}{2 k + 1}
                    + \cdots \right)$ \\
  \hline
\end{tabular}
\end{showexample}
%
The classical workarounds \cmd{\textbackslash}\oarg{dimen} and
\cmd{noalign}\tex{\{}\cmd{vspace}\marg{dimen}\tex{\}} are not very powerful
and need fine tuning.

\section{How to solve this?}

The \pack{cellspace} loads several packages to carry out its job:
\pack{array}, \pack{ifthen}, and~\pack{calc} (along \tex{amsmath} when
the option \tex{math} is passed to \tex{cellspace}).  It redefines several
internals, still not too many (\cmd{@startpbox} and \cmd{@endpbox}).  By
default, the tables will behave as usual.  To improve the spacing of your
tables, you must change the table preamble and prepend \tex{S} to the
column types \tex{l}, \tex{c}, \tex{r}.  The same holds for the paragraph
columns \tex{p}, \tex{m}, and~\tex{b}, except that they must be surrounded
by an extra pair of braces.  For instance, the default behaviour of a
table beginning with
%
\begin{verbatim}
\begin{tabular}{l l l p{3cm}}
\end{verbatim}
%
should be changed into
%
\begin{verbatim}
\begin{tabular}{Sl Sl Sl S{p{3cm}}}
\end{verbatim}

The \pack{cellspace} has two parameters governing the spacing of the
cells.  The dimension \cmd{cellspacetoplimit} is the minimal spacing
required between the actual text the cell is made up of and the top of the
cell (where \cmd{hline}s may appear), if the spacing is less than this
threshold, a space of \cmd{cellspacetoplimit} is added, otherwise nothing
is done. Conversely, at the bottom the dimension
\cmd{cellspacebottomlimit} is the minimal space required between the
bottom of the cell and the text itself. This parameters can be changed in
the document preamble using \cmd{setlength}.  Together with
\pack{booktabs}, the results look rather good.  \pack{cellspace} may work
with other column types (defined through \cmd{newcolumntype} for
instance); it will assume by default that an unknown column is a LR-mode
column.  Otherwise you have to tell \pack{cellspace} something like
%
\begin{verbatim}
\addparagraphcolumntypes{X}
\end{verbatim}
%
(The previous example is needed to make \pack{tabularx} work with
\pack{cellspace}.)

The following examples use a setting of $2\,\mathrm{pt}$ for both top and
bottom space limits.
%
\compare{%
  \begin{tabular}[t]{cc}
    \hline
    \itshape Bond &
    \itshape Distance ($\mathrm{\mathring{A}})$ \\
    \hline
    C--C          & $1.53$ \\
    C--H          & $1.10$ \\
    \hline
  \end{tabular}
}{%
  \begin{tabular}[t]{Sc Sc}
    \hline
    \itshape Bond &
    \itshape Distance ($\mathrm{\mathring{A}})$ \\
    \hline
    C--C          & $1.53$ \\
    C--H          & $1.10$ \\
    \hline
  \end{tabular}
}
%
Some space is added at the top the first row because it doesn't fulfil
the condition imposed by the package.  Similarly some space is added at
the bottom of the first row.  It has to be noted, however, that no space
is added in other rows because they are far enough from the cell
separations.

\compare{%
  \begin{tabular}[t]{c}
    \hline
    \itshape Formula \\
    \hline
    $\displaystyle
    e = 1 + \frac{1}{2} + \frac{1}{6}  + \cdots
                         + \frac{1}{k!} + \cdots $ \\
    $\displaystyle
     \pi = 4 \left( 1 - \frac{1}{3} + \frac{1}{5} + \cdots
                      + \frac{(-1)^k}{2 k + 1}
                      + \cdots \right)$ \\
    \hline
  \end{tabular}
}{%
  \begin{tabular}[t]{Sc}
    \hline
    \itshape Formula \\
    \hline
    $\displaystyle
    e = 1 + \frac{1}{2} + \frac{1}{6}  + \cdots
                         + \frac{1}{k!} + \cdots $ \\
    $\displaystyle
     \pi = 4 \left( 1 - \frac{1}{3} + \frac{1}{5} + \cdots
                      + \frac{(-1)^k}{2 k + 1}
                      + \cdots \right)$ \\
    \hline
  \end{tabular}
}
%
Here it can be immediately remarked that the `Formula' row has the same
height in both examples.  Only the formula lines are affected.  The best
layout is obtained with \pack{booktabs}:
%
\compare{%
  \begin{tabular}[t]{@{} c @{}}
    \toprule
    \itshape Formula \\
    \midrule
    $\displaystyle
    e = 1 + \frac{1}{2} + \frac{1}{6}  + \cdots
                         + \frac{1}{k!} + \cdots $ \\
    $\displaystyle
     \pi = 4 \left( 1 - \frac{1}{3} + \frac{1}{5} + \cdots
                      + \frac{(-1)^k}{2 k + 1}
                      + \cdots \right)$ \\
    \bottomrule
  \end{tabular}
}{%
  \begin{tabular}[t]{@{} Sc @{}}
    \toprule
    \itshape Formula \\
    \midrule
    $\displaystyle
    e = 1 + \frac{1}{2} + \frac{1}{6}  + \cdots
                         + \frac{1}{k!} + \cdots $ \\
    $\displaystyle
     \pi = 4 \left( 1 - \frac{1}{3} + \frac{1}{5} + \cdots
                      + \frac{(-1)^k}{2 k + 1}
                      + \cdots \right)$ \\
    \bottomrule
  \end{tabular}
}

The following example shows that how it works with paragraph columns.
%
\compare{%
  \begin{tabular}[t]{p{5cm}}
    \hline
    $\displaystyle \frac{1}{2}$
    A A A A A A A A A A A A A A
    A A A A A A A A A A A A A A
    A A A A A A A A A A A A A A
    A A A A A A A A A A A A A A
    A A A A A A A A A A A A A A
    $\displaystyle \frac{1}{2}$ \\
    \hline
  \end{tabular}
}{%
  \begin{tabular}[t]{S{p{5cm}}}
    \hline
    $\displaystyle \frac{1}{2}$
    A A A A A A A A A A A A A A
    A A A A A A A A A A A A A A
    A A A A A A A A A A A A A A
    A A A A A A A A A A A A A A
    A A A A A A A A A A A A A A
    $\displaystyle \frac{1}{2}$ \\
    \hline
  \end{tabular}
}

\section{Matrices}

The package can be loaded with the options \tex{nomath} (default) and
\tex{math}.  The latter does two things: it first loads the
\tex{amsmath} package and then redefines
the command that is invoked when one typesets matrices
(\tex{\textbackslash env@matrix} --- this improvement was suggested by
Bastien Roucaries).  The following
examples illustrate what happens with (left) or without (right)
\pack{cellspace}'s correction in a \tex{pmatrix} environment.
%
\[
\begingroup
\makeatletter
  \def\env@matrix{\hskip -\arraycolsep
  \let\@ifnextchar\new@ifnextchar
  \array{*\c@MaxMatrixCols c}}
\makeatother
%
\text{\tex{nomath} option\ }%
\begin{pmatrix}
+\frac{1}{2} & +\frac{1}{3} \\
-\frac{1}{3} & +\frac{1}{2} \\
\end{pmatrix}
\endgroup
\qquad
\text{\tex{math} option\ }%
\begin{pmatrix}
+\frac{1}{2} & +\frac{1}{3} \\
-\frac{1}{3} & +\frac{1}{2} \\
\end{pmatrix}
\]

As a~consequence, the \tex{amsmath} package can be loaded beforehand
with other packages (such as \tex{empheq} or \tex{mathtools}), were an
incompatibility to arise from one's loading it later.

\section{Using a different column name}

An other column name than \tex{S} can be used if \tex{S} conflicts
with the user's own definitions or a package's.  A keyval option can
be passed to the package to that effect;
%
\begin{verbatim}
\usepackage[column=O]{cellspace}
\end{verbatim}
%
can be used to provide cellspace's functionality through the \tex{O}
column modifier instead of the \tex{S} column modifier.


\section{Bugs and limitations}

This package hasn't been heavily tested, so there may be plenty of bugs.
As usual, bugs will certainly arise in complicated situations because
tables can become very complicated (for instance, this package hasn't been
designed with nested tables into mind).  In simple cases however, it should
work nicely.  The package loads \pack{array} ensuring a peaceful
collaboration with each other.  Other packages of special interest haven't
been tested: \pack{tabularx}\footnote{A one minute test seems to indicate
  that it works provided \pack{tabularx} is loaded after \pack{cellspace}.
  The example on this page illustrates this.  A thirty second test
indicates that \pack{cellspace} works correctly with \pack{longtable} as
long as only LR-mode columns are concerned.} and \pack{longtable} in
particular.
%
\begin{trivlist}\item
\begin{tabularx}{\linewidth}{l SX}
  \hline
  A       & B B B B B B B B \\
  A A     & B B B B B B B B
            B B B B B B B B \\
  A A A A & $\dfrac{1}{2}$B B B B B B B B
            B B B B B B B B
            B B B B B B B B
            B B B B B B B B$\dfrac{1}{2}$ \\
  \hline
\end{tabularx}
\end{trivlist}

\pack{tabls}\footnote{Thanks to Jim Hefferon for pointing this out.} is a
package that does a similar job as \pack{cellspace}.
The differences are: \pack{tabls} affects a whole table while
\pack{cellspace} affects only the columns on which applies the modifier
\tex{S}.  Each cell affected by \pack{cellspace} is affected in the same
way: there must be enough space above ($>\cmd{cellspacetoplimit}$) and
below the text ($>\cmd{cellspacebottomlimit}$), even if no \cmd{hline} is
present.  In the other hand, \pack{tabls} ensures a minimal distance using
one threshold parameter \cmd{tablinesep} applying between rows not
separated by a rule and using a systematic spacer \cmd{arraylinesep} when
there is a rule (whereas \pack{cellspace} can leave a cell as is if the
distance to the rule is big enough).  As a last point, \pack{tabls}, as
far as I can see, cannot work with \pack{array}.

\section{What's new}

\begin{description}
\item [v1.8.1] Package becomes aware of the latest version of the
      array package
\end{description}

\end{document}
